\section{The Internet}

\noindent
Terminology and concepts of the internet, which will be used throughout this text.

\begin{Def}[Internet]

    The \textbf{Internet} is a global network of distributed system communicating over an \textbf{Internet Protocol} (IP) \cite{cloudflare_internet_protocol}.
    Documents served over the internet are referred to as \textbf{webpages} or \textbf{websites}.
\end{Def}
\begin{Def}[HTTP \& HTML]

    \textbf{HTTP} (HyperText Transfer Protocol), the protocol which transfer data over the internet, 
    distributing \textbf{HTML} (HyperText Markup Language) documents. Such 
    documents include \textbf{hyperlinks} to other websites, images, and other media \cite{rfc9110}.
\end{Def}
\begin{Def}[RFC (Request for Comments)]

    \textbf{RFC} (Request for Comments) is a publication from the \textbf{Internet Engineering Task Force} (IETF) 
    and the \textbf{Internet Society} (ISOC). This body governs the specifications for the internet and its protocols \cite{rfc}.
\end{Def}

\begin{Def}[DNS and IP Addresses]

    An \textbf{Internet Protocol address} (IP address) is a unique identifier for a device on a network. 
    The \textbf{Domain Name System} (DNS) maps domain names to IP addresses \cite{rfc760}.
\end{Def}

\newpage

\begin{Def}[Web Browser]

    A \textbf{web browser} is a software application for accessing the \textbf{World Wide Web} (WWW) \cite{ou_internet_history}.
\end{Def}

\begin{Def}[URL (Uniform Resource Locator)]

    A \textbf{URL} (Uniform Resource Locator) references each webpage, specifying protocol, domain, and path \cite{w3c_html_href_draft}.
    E.g., \texttt{http://www.example.com/path/to/resource}.
    \begin{itemize}
        \item \textbf{Protocol}: \texttt{http}
        \item \textbf{Domain}: \texttt{www.example.com}
        \item \textbf{Path}: \texttt{/path/to/resource}
    \end{itemize}
\end{Def}
\begin{Def}[Client-Server Model]

    Most of the internet operates on a \textbf{client-server model}, where an agent device\textendash the \textbf{client}\textendash requests data from another agent\textendash the \textbf{server}\textendash 
    which serves an appropriate response. Clients are not servers and vice versa, as they receive and interpret data differently \cite{cloudflare_client_server}.
\end{Def}

\begin{Def}[HTTP Methods]
    
        When a client makes a request to a server, they must specify their intent, categorized by \textbf{HTTP methods} \cite{rfc2616}:
        \begin{itemize}
            \item \textbf{GET}: Retrieve data from the server.
            \item \textbf{POST}: Send data to the server.
            \item \textbf{PUT}: Update data on the server.
            \item \textbf{DELETE}: Remove data from the server.
        \end{itemize}
\end{Def}
\begin{Def}[HTTP Headers]

    \textbf{HTTP headers} are key-value pairs sent between the client and server to provide \textbf{metadata} about the request or response.
    \textbf{Metadata} is data about the transmitted data, telling the receiver how the incoming data should be interpreted \cite{rfc2616}.
\end{Def}


\newpage
\noindent
Tim Berners-Lee and his team at CERN developed the first web server and browser in 1989 \cite{w3c_http_history}.

\begin{table}[h!]
    \centering
    \begin{tabular}{@{}p{3cm}p{10cm}@{}}
    \toprule
    \textbf{HTTP Version} & \textbf{Description} \\ \midrule
    HTTP/0.9 (1991)       & Only supports GET method (retrieving HTML alone). \\
    HTTP/1.0 (1996)       & RFC\#1945, adding support for metadata in HTTP headers, status codes, and POST and HEAD methods \cite{rfc1945}. \\
    HTTP/1.1 (1997)       & Defined in RFC\#2068 and later updated by RFC\#2616, introduced persistent connections, chunked transfer encoding, and additional cache control mechanisms \cite{rfc2068}\cite{rfc2616}. \\
    HTTP/2 (2015)         & RFC\#7540, improving performance by enabling request and response multiplexing, header compression, and prioritization \cite{rfc7540}. \\
    HTTP/3 (2022)         & Builds upon HTTP/2's features and uses the QUIC transport protocol to reduce latency and improve security. \cite{rfc9114} \\ \bottomrule
    \end{tabular}
    \caption{Evolution of HTTP Versions}    
    \label{tab:http_versions}
\end{table}

\begin{Note}
    \textbf{Note:} In short, \textbf{Persistent Connections} allow multiple requests and responses to be sent over a single connection, reducing latency and improving performance \cite{rfc2616}.
    \textbf{Chunked Transfer Encoding} allows the server to send data in chunks, enabling the client to start processing data before the entire response is received \cite{rfc2616}.
    \textbf{Multiplexing}, is the ability to send multiple requests and responses over a single connection, reducing latency and improving performance \cite{multiplexing_networkencyclopedia}. We 
    will discuss \textbf{QUIC} and other transfer protocols in a later section.

\end{Note}

\section{Data Transmission}


