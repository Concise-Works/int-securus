\usepackage[dvipsnames]{xcolor} % Required for specifying custom colors
\usepackage[showframe=false]{geometry} % Required for adjusting page dimensions and margins
\usepackage[colorlinks=true, linkcolor=blue, urlcolor=blue]{hyperref}
% Packages for custom shapes and boxes
\usepackage{tikz} % Required for drawing custom shapes
\usepackage[framemethod=TikZ]{mdframed} % Required for creating the theorem, definition, exercise and corollary boxes
% \newcounter{theo}[section]\setcounter{theo}{0}

% THEOREM BOXES
\newcounter{theo}[section]\setcounter{theo}{0}
\renewcommand{\thetheo}{\arabic{section}.\arabic{theo}}

\newenvironment{theo}[2][]{%
    \refstepcounter{theo}

    \ifstrempty{#1}%
    % if condition (without title)
    {\mdfsetup{%
            frametitle={%
                    \tikz[baseline=(current bounding box.east),outer sep=0pt]
                    \node[anchor=east,rectangle,fill=black,text=white]
                    {\strut Theorem~\thetheo};}
        }%
        % else condition (with title)
    }{\mdfsetup{%
            frametitle={%
                    \tikz[baseline=(current bounding box.east),outer sep=0pt]
                    \node[anchor=east,rectangle,fill=black,text=white]
                    {\strut Theorem~\thetheo:~#1};}%
        }%
    }%
    % Both conditions
    \mdfsetup{%
        innertopmargin=6pt,innerbottommargin=12pt,linecolor=black,%
        linewidth=2pt,topline=true,%
        frametitleaboveskip=\dimexpr-\ht\strutbox\relax%
    }

    \begin{mdframed}[]\relax}{
    \end{mdframed}}

% DEFINITION BOXES
\newcounter{Def}[section]\setcounter{Def}{0}
\renewcommand{\theDef}{\arabic{section}.\arabic{Def}}

\newenvironment{Def}[2][]{%
    \refstepcounter{Def}

    \ifstrempty{#1}%
    % if condition (without title)
    {\mdfsetup{%
            frametitle={%
                    \tikz[baseline=(current bounding box.east),outer sep=0pt]
                    \node[anchor=east,rectangle,fill=black,text=white]
                    {\strut Definition~\theDef};}
        }%
        % else condition (with title)
    }{\mdfsetup{%
            frametitle={%
                    \tikz[baseline=(current bounding box.east),outer sep=0pt]
                    \node[anchor=east,rectangle,fill=black,text=white]
                    {\strut Definition~\theDef:~#1};}%
        }%
    }%
    % Both conditions
    \mdfsetup{%
        innertopmargin=6pt,
        innerbottommargin=12pt,
        linecolor=black,%
        linewidth=2pt,topline=true,%
        frametitleaboveskip=\dimexpr-\ht\strutbox\relax%
    }

    \begin{mdframed}[]\relax}{%
    \end{mdframed}}


% TIP BOXES
\newcounter{Tip}[section]\setcounter{Tip}{0}
\renewcommand{\theTip}{\arabic{section}.\arabic{Tip}}

\newenvironment{Tip}{%
    \refstepcounter{Tip}
    \mdfsetup{%
        backgroundcolor=OliveGreen!10,
        innertopmargin=12pt,
        innerbottommargin=12pt,
        linecolor=black,
        linewidth=0pt,
        topline=false,
        frametitleaboveskip=\dimexpr-\ht\strutbox\relax
    }
    \begin{mdframed}[]\relax
        \textbf{Tip:}
        }{%
    \end{mdframed}
}

% GREEN BOXES
\newenvironment{greenbox}{%
    \mdfsetup{%
        backgroundcolor=OliveGreen!10,
        innertopmargin=12pt,
        innerbottommargin=12pt,
        linecolor=black,
        linewidth=0pt,
        topline=false,
        frametitleaboveskip=\dimexpr-\ht\strutbox\relax
    }
    \begin{mdframed}[]\relax

        }{%
    \end{mdframed}
}

% GRAY BOXES
\newenvironment{graybox}{%
    \mdfsetup{%
        backgroundcolor=black!10,
        innertopmargin=12pt,
        innerbottommargin=12pt,
        linecolor=black,
        linewidth=0pt,
        topline=false,
        frametitleaboveskip=\dimexpr-\ht\strutbox\relax
    }
    \begin{mdframed}[]\relax

        }{%
    \end{mdframed}
}

% NOTE BOXES
\newcounter{Note}[section]\setcounter{Note}{0}
\renewcommand{\theNote}{\arabic{section}.\arabic{Note}}

\newenvironment{Note}{%
    \refstepcounter{Note}
    \mdfsetup{%
        backgroundcolor=black!10,
        innertopmargin=10pt,
        innerbottommargin=8pt,
        linecolor=black,
        linewidth=0pt,
        topline=false,
        frametitleaboveskip=\dimexpr-\ht\strutbox\relax
    }
    \begin{mdframed}[]\relax
        }{%
    \end{mdframed}
}

% GENERIC BOXES

\newenvironment{gbox}{%
    \mdfsetup{%
        backgroundcolor=white,
        innertopmargin=12pt,
        innerbottommargin=6pt,
        linecolor=black,
        linewidth=1pt,
        topline=true,
        frametitleaboveskip=\dimexpr-\ht\strutbox\relax
    }
    \begin{mdframed}[]\relax
        % Your content goes here
        }{%
    \end{mdframed}
}

% PROOF BOXES

% \newcounter{theo}[section]\setcounter{theo}{0}

% PROOF BOXES

\newcounter{Proof}[section]\setcounter{Proof}{0}
\renewcommand{\theProof}{\arabic{section}.\arabic{Proof}}

\newenvironment{Proof}[1][]{%
    \refstepcounter{Proof}

    \ifstrempty{#1}%
    % if condition (without title)
    {\mdfsetup{%
            frametitle={%
                    \tikz[baseline=(current bounding box.east),outer sep=0pt]
                    \node[anchor=east,rectangle,fill=white,text=black]
                    {\strut Proof~\theProof};}
        }%
        % else condition (with title)
    }{\mdfsetup{%
            frametitle={%
                    \tikz[baseline=(current bounding box.east),outer sep=0pt]
                    \node[anchor=east,rectangle,fill=white,text=black]
                    {\strut Proof~\theProof:~#1};}%
        }%
    }%
    % Both conditions
    \mdfsetup{%
        innertopmargin=6pt,
        innerbottommargin=12pt,
        linecolor=black,%
        linewidth=1pt,topline=true,%
        frametitleaboveskip=\dimexpr-\ht\strutbox\relax%
    }

    \begin{mdframed}[]\relax}{
        \hfill$\blacksquare$
    \end{mdframed}
}

% EXAMPLE BOXES

\newcounter{Example}[section]\setcounter{Example}{0}
\renewcommand{\theExample}{\arabic{section}.\arabic{Example}}

\newenvironment{Example}[1][]{%
    \refstepcounter{Example}

    \ifstrempty{#1}%
    % if condition (without title)
    {\mdfsetup{%
            frametitle={%
                    \tikz[baseline=(current bounding box.east),outer sep=0pt]
                    \node[anchor=east,rectangle,fill=white,text=black]
                    {\strut Example~\theExample};}
        }%
        % else condition (with title)
    }{\mdfsetup{%
            frametitle={%
                    \tikz[baseline=(current bounding box.east),outer sep=0pt]
                    \node[anchor=east,rectangle,fill=white,text=black]
                    {\strut Example~\theExample:~#1};}%
        }%
    }%
    % Both conditions
    \mdfsetup{%
        innertopmargin=6pt,
        innerbottommargin=12pt,
        linecolor=black,%
        linewidth=1pt,topline=true,%
        frametitleaboveskip=\dimexpr-\ht\strutbox\relax%
    }

    \begin{mdframed}[]\relax}{
        \hfill$\blacksquare$
    \end{mdframed}
}

% Function Counter
\newcounter{Func}[section]
\renewcommand{\theFunc}{\arabic{section}.\arabic{Func}}

% Function Environment
\newenvironment{Func}[1][]{
    \refstepcounter{Func}
    \ifstrempty{#1}% 
    {% If no title is provided
        \mdfsetup{
            frametitle={
                \tikz[baseline=(current bounding box.east),outer sep=0pt]
                \node[anchor=east,rectangle,fill=black,text=white]
                {\strut Function~\theFunc};}
        }
    }{% If a title is provided
        \mdfsetup{
            frametitle={
                \tikz[baseline=(current bounding box.east),outer sep=0pt]
                \node[anchor=east,rectangle,fill=black,text=white]
                {\strut Function~\theFunc:~#1};}
        }
    }
    % Common settings for both cases
    \mdfsetup{
        innertopmargin=10pt,
        innerbottommargin=10pt,
        skipabove=\baselineskip,
        skipbelow=\baselineskip,
        linecolor=black,
        linewidth=1pt,
        topline=true,
        frametitleaboveskip=\dimexpr-\ht\strutbox\relax,
        frametitlealignment=\raggedright
    }
    \begin{mdframed}
}{
    \end{mdframed}
}

 % This file contains the code for the boxes

% Packages for math symbols and images
\usepackage{amssymb} % Math symbols such as \mathbb
\usepackage{amsmath} % Required for \begin{align}...\end{align}
\DeclareMathOperator{\lcm}{lcm}
\usepackage{mathtools}
\DeclarePairedDelimiter\ceil{\lceil}{\rceil}
\DeclarePairedDelimiter\floor{\lfloor}{\rfloor}
\usepackage{graphicx} % Required for including images

% Tables
\usepackage{multirow} % Required for creating multirow cells in tables
\usepackage{array} % Required for creating tables
\usepackage{colortbl} % Add this line to import the colortbl package
\setlength{\tabcolsep}{18pt} % Default value: 6pt
\renewcommand{\arraystretch}{1.5} % Default value: 1
\usepackage{changepage} % Required for adjusting the width of the table


% Package for customizing captions
\usepackage{caption}  % Required for customizing captions

\usepackage{setspace} % Required for adjusting line spacing
\usepackage{etoolbox} % For \ifstrempty

\usepackage{subcaption} % for subfigures

\usepackage[linesnumbered, noend]{algorithm2e}
\usepackage{algpseudocode}
\usepackage{makecell} % for line break in table cell

\usepackage{longdivision}% Package for long division algorithm

% commands

% Tikz
% \tikzset{
%     pics/machine/.style={
%             code={
%                     \coordinate (-in) at ({-0.5*\pgfkeysvalueof{/tikz/machine/width}},0);
%                     \coordinate (-out) at ({0.5*\pgfkeysvalueof{/tikz/machine/width}},0);
%                     \coordinate (-north) at (0,{0.5*\pgfkeysvalueof{/tikz/machine/height}});
%                     \coordinate (-south) at (0,{-0.5*\pgfkeysvalueof{/tikz/machine/height}});
%                     \path[/tikz/machine/filling]
%                     ([shift={(-0.75em,0.5em)}]-in)
%                     -- ([shift={(0,0.25em)}]-in)
%                     -- ([shift={(0,-5pt)}]-north -| -in)
%                     arc[start angle=180, end angle=90, radius=5pt]
%                     -- ([shift={(-5pt,0)}]-north -| -out)
%                     arc[start angle=90, end angle=0, radius=5pt]
%                     -- ([shift={(0,0.25em)}]-out)
%                     -- ([shift={(0.75em,0.5em)}]-out)
%                     -- ([shift={(0.75em,-0.5em)}]-out)
%                     -- ([shift={(0,-0.25em)}]-out)
%                     -- ([shift={(0,5pt)}]-south -| -out)
%                     arc[start angle=360, end angle=270, radius=5pt]
%                     -- ([shift={(5pt,0)}]-south -| -in)
%                     arc[start angle=270, end angle=180, radius=5pt]
%                     -- ([shift={(0,-0.25em)}]-in)
%                     -- ([shift={(-0.75em,-0.5em)}]-in)
%                     -- cycle;
%                     \draw[/tikz/machine/border]
%                     ([shift={(-0.75em,0.5em)}]-in)
%                     -- ([shift={(0,0.25em)}]-in)
%                     -- ([shift={(0,-5pt)}]-north -| -in)
%                     arc[start angle=180, end angle=90, radius=5pt]
%                     -- ([shift={(-5pt,0)}]-north -| -out)
%                     arc[start angle=90, end angle=0, radius=5pt]
%                     -- ([shift={(0,0.25em)}]-out)
%                     -- ([shift={(0.75em,0.5em)}]-out);
%                     \draw[/tikz/machine/border]
%                     ([shift={(0.75em,-0.5em)}]-out)
%                     -- ([shift={(0,-0.25em)}]-out)
%                     -- ([shift={(0,5pt)}]-south -| -out)
%                     arc[start angle=360, end angle=270, radius=5pt]
%                     -- ([shift={(5pt,0)}]-south -| -in)
%                     arc[start angle=270, end angle=180, radius=5pt]
%                     -- ([shift={(0,-0.25em)}]-in)
%                     -- ([shift={(-0.75em,-0.5em)}]-in);
%                     \node (-node) at (0,0) {#1};
%                 }
%         },
%     machine/width/.initial={10.5em},
%     machine/height/.initial={5em},
%     machine/filling/.style={
%             left color=cyan!50!OliveGreen!25,
%             right color=cyan!50!OliveGreen!25,
%             middle color=cyan!50!OliveGreen!10
%         },
%     machine/border/.style={
%             OliveGreen!50!cyan
%         }
% }

% Color Text
\definecolor{BlueText}{RGB}{0,51,204}
\newcommand{\bt}[1]{\textcolor{BlueText}{#1}}

\newcommand{\Div}{%
    \par\noindent\hrulefill\par
}

\renewcommand{\clearforchapter}{ \newpage}

\makeatletter
\NewCommandCopy\@@pmod\pmod
\DeclareRobustCommand{\pmod}{\@ifstar\@pmods\@@pmod}
\def\@pmods#1{\mkern4mu({\operator@font mod}\mkern 6mu#1)}
\makeatother



\newcolumntype{g}{>{\columncolor{OliveGreen!10}}c}

\newcommand{\qres}{(\mathbb{Z}_p^*)^2}
\newcommand{\ires}{(\mathbb{Z}_n^*)^2}
\newcommand{\pres}{\mathbb{Z}_p^*}
\newcommand{\peres}{\mathbb{Z}_{p^{e}}^*}
\newcommand{\gres}{\mathbb{Z}_n}
\newcommand{\Z}{\mathbb{Z}}


\algnewcommand\algorithmicforeach{\textbf{for each}}
\algdef{S}[FOR]{ForEach}[1]{\algorithmicforeach\ #1\ \algorithmicdo}

\newcommand*{\carry}[1][1]{\overset{#1}}
\newcolumntype{B}[1]{r*{#1}{@{\,}r}}